%%
%% This is file `sample-manuscript.tex',
%% generated with the docstrip utility.
%%
%% The original source files were:
%%
%% samples.dtx  (with options: `manuscript')
%% 
%% IMPORTANT NOTICE:
%% 
%% For the copyright see the source file.
%% 
%% Any modified versions of this file must be renamed
%% with new filenames distinct from sample-manuscript.tex.
%% 
%% For distribution of the original source see the terms
%% for copying and modification in the file samples.dtx.
%% 
%% This generated file may be distributed as long as the
%% original source files, as listed above, are part of the
%% same distribution. (The sources need not necessarily be
%% in the same archive or directory.)
%%
%% The first command in your LaTeX source must be the \documentclass command.
%%%% Small single column format, used for CIE, CSUR, DTRAP, JACM, JDIQ, JEA, JERIC, JETC, PACMCGIT, TAAS, TACCESS, TACO, TALG, TALLIP (formerly TALIP), TCPS, TDSCI, TEAC, TECS, TELO, THRI, TIIS, TIOT, TISSEC, TIST, TKDD, TMIS, TOCE, TOCHI, TOCL, TOCS, TOCT, TODAES, TODS, TOIS, TOIT, TOMACS, TOMM (formerly TOMCCAP), TOMPECS, TOMS, TOPC, TOPLAS, TOPS, TOS, TOSEM, TOSN, TQC, TRETS, TSAS, TSC, TSLP, TWEB.
% \documentclass[acmsmall]{acmart}

%%%% Large single column format, used for IMWUT, JOCCH, PACMPL, POMACS, TAP, PACMHCI
\documentclass[acmlarge,screen]{acmart}

%%%% Large double column format, used for TOG
%\documentclass[acmtog, authorversion]{acmart}

%%%% Generic manuscript mode


%%
%% \BibTeX command to typeset BibTeX logo in the docs
\AtBeginDocument{%
  \providecommand\BibTeX{{%
    \normalfont B\kern-0.5em{\scshape i\kern-0.25em b}\kern-0.8em\TeX}}}


%%
%% The majority of ACM publications use numbered citations and
%% references.  The command \citestyle{authoryear} switches to the
%% "author year" style.
%%
%% If you are preparing content for an event
%% sponsored by ACM SIGGRAPH, you must use the "author year" style of
%% citations and references.
%% Uncommenting
%% the next command will enable that style.
%%\citestyle{acmauthoryear}

%%
%% end of the preamble, start of the body of the document source.
\begin{document}

%%
%% The "title" command has an optional parameter,
%% allowing the author to define a "short title" to be used in page headers.
\title{UVU MCS Graduate Paper}

%%
%% The "author" command and its associated commands are used to define
%% the authors and their affiliations.
%% Of note is the shared affiliation of the first two authors, and the
%% "authornote" and "authornotemark" commands
%% used to denote shared contribution to the research.

\author{Benjamin Stoneking}
\affiliation{%
 \institution{Candidate}
}
\email{benjamin.stoneking@protonmail.com}

\author{Frank Jones}
\affiliation{%
 \institution{Advisor}
}
\email{frankj@uvu.edu} 

%%
%% By default, the full list of authors will be used in the page
%% headers. Often, this list is too long, and will overlap
%% other information printed in the page headers. This command allows
%% the author to define a more concise list
%% of authors' names for this purpose.
\renewcommand{\shortauthors}{Candidate First Last}

%%
%% The abstract is a short summary of the work to be presented in the
%% article.
\begin{abstract}
	The project explores the process of implementing an embedded real-time digital audio synthesizer. Emphasis is placed on leveraging the inherent strengths of the hardware (timers, hardware interrupts, etc) to produce a reliable musical instrument. The project is a focused exercise in implementing a digital audio processing pipeline with software implementations of common components such as oscillators, pitch control, signal gain attenuation, and filtering and it illustrates the process of implementing a capable, complex and extendable embedded system on a platform with limited processing power. The final product is a fundamental subtractive synthesizer: It responds to input from generic MIDI devices to produce an audio signal. Volume dynamics and timbre control are added to the signal with envelopes and filters. The device has a hardware interface consisting of knobs, encoders, and a screen allowing direct real-time manipulation of the processing. A PC application was implemented to control the system remotely. The system is robust, but many features remain to be implemented. Implementing the expected functionality of electronic musical instruments was challenging. The project code base is close to 2250 lines. Substantial time was spent writing drivers to interface with hardware which often added up to very few lines of code. Processing speed and instruction space is limited and large code bases can quickly outgrow the capabilities of the device requiring efficient and concise code. Substantial effort and time was required to prototype the hardware interface on a breadboard and revise as new components were integrated. More time was required to design and build a permanent hardware fixture on a circuit board.
\end{abstract}

%%
%% Keywords. The author(s) should pick words that accurately describe
%% the work being presented. Separate the keywords with commas.
\keywords{audio, synthesis, embedded, microcontroller, real-time, midi, hardware, oscillator, filter, envelope, DSP}

%%
%% This command processes the author and affiliation and title
%% information and builds the first part of the formatted document.
\maketitle

\section{Introduction}
\subsection{Purpose}
	The acquisition of musical gear is a common obsession that plagues millions of musicians around the world. The cost of gear for performance or music production presents a major financial struggle to musicians who already must work within a tight budget. A common thought occurs to musicians: "What if I just make my own X? Maybe I can save some money." This discussion occurs frequently within online and offline forums. Often times, musicians who have experience in making their own digital or analog instruments will give the same answer: "Yes. You can just make your own X. But no. You won't save any money. You will spend more and you should just save up some cash to buy your gear from a compony like Roland, Yamaha, Korg, etc." Over the years, I've researched this very topic and found this conclusion to be true but with a small caveat: You can in fact save money by making your audio production tool as a \textbf{digital system}. You can write as much software as you want. You can refactor and rebuild you software tool as much as you want, and never spend a dime. When physical electronic components are removed from the equation, if you have a system to run your software, you can make all your music production gear for free. This caveat has a caveat of its own: This takes substaintial time. Time is money. Therefore, every hour spent in implementing your system is implicitly adding to the price tag of your system.
	
	I've known these facts for a long time and have seen them in action as my education has had me walk through the process of making \textit{yet another} database management system, and \textit{yet another} virtual machine. Unsurprisingly, my database hasn't replaced MySQL and my virtual machine has not replaced the Java Virtual Machine. Reinventing the proverbial wheel did not make the effort fruitless. Valuable wisdom and knowledge of how computers work was gained from these experiences. I applied this same principle to my experience as a musician with the goal of my master's project: Make \textit{yet another} digital synthesizer on an embedded microcontroller. \textbf{The endgame of this project was the journey itself: to learn how to implement a complex real-time embedded system, develop a digital audio pipeline, make a hardware interface to control it and see what invaluable knowledge can be extracted from the experience.} There is also the intrinsic side benefit of indulging my personal fantasy of making a piece of musical gear that is wholly mine. The aim of this project was to accomplish the task of creating this device and documenting the process and lessons learned: What is a valid working architecture to making such a system? Where does one begin? What kind of failure and setbacks should be expected? Can it be done within a reasonable amount of time without substantial reliance on previously existing frameworks and libraries? What sort of mathematical and technical knowledge is required?

\subsection{Project Criteria}

	\subsubsection{Minimum Requirements}
	In order to meet the definition of a hardware subtractive synthesizer it must meet the following minimum requirements (see Concepts for definitions):
	\begin{enumerate}
		\item The device must produce audio at a minimum quality of 44.1khz and 16 bit-depth (CD Audio Quality)
		\item Notes can be played by way of any generic MIDI keyboard.
		\item An \textbf{oscillator} component must be implemented. It must generate the most common wave forms: Sin, sawtooth, square, triangle and white noise
		\item The synthesizer (hereafter referred to as synth), but be polyphonic, capable of playing up to 8 discrete notes at a time.
		\item A \textbf{voice} component allowing mixing of multiple oscillator waveforms must be implemented.
		\item An \textbf{envelope} component must be implemented to dynamically attenuate the gain of each voice over time.
		\item A \textbf{filter} component must be implemented to be able to selectively cut frequencies from the final signal. It should support lowpass, highpass and bandpass filtering. Filter frequency cutoff and resonance must be adjustable in real-time.
		\item A \textbf{low-frequency oscillator (LFO)} component must be implemented which the user may configure to modulate arbitrary signals within the system.
		\item Custom "patch" saving. Users can save their current configuration to a patch bank, and recall their saved a previous.
		\item A \textbf{hardware interface} allowing personal interaction between the user and the system must be implemented. A MIDI input jack must be included for plugging in a keyboard. A 1/8" audio output jack should allow the user to hear playback with headphones or speakers. It should have a series of knobs and a screen communicate the system state to the user. The knobs should allow the user to navigate a menu to select a variety of parameters they wish to manipulate. The knobs should adapt to the context of the menu to adjust parameters appropriately.
	\end{enumerate}

	\subsubsection{Additional Features}
	Mentioned above is the absolute bare minimum representation of what would constitute a common subtractive synthesizer. A series of other common but not required features were planned for the project:
	\begin{itemize}
		\item Additional generic envelope for modulating arbitrary signals on system
		\item MIDI Pitch and Modulation wheel support
		\item Additional audio FX: Delay, reverb, overdrive, etc.
		\item 3D printed enclosure box
		\item Expand synthesis options: Frequency Modulation (FM Synthesis), Sample based synthesis
	\end{itemize}
	
	\subsubsection{Constraints}
	To maximize the learning potential of this project and focus on the process rather than the final destination, I decided to set a few restrictions:
	\begin{enumerate}
		\item Language: The firmware must be implemented in a low-level programming language such as C/C++.
		\item Libraries: Only the standard C/C++ libraries may be used. A Hardware Abstraction Layer to minimize hardware peripheral set up time may be used. Other than that exception, all software must be written from scratch. This includes digital signal processing (DSP) libraries; However, every audio processing function is handcrafted.
		\item References: Referencing and reading other implementations is not explicitly restricted. Every effort must be put forth to understand the mathematical concepts of DSP and implement them without reliance on existing implementations.
	\end{enumerate}
	
\subsection{Results}
	The final state of project met most of the minimum requirements but failed to fulfill every feature: There was not enough time to implement LFOs info the system, and HAL library issues involving writing and reading from flash memory space got in the way of fully implementing a user patch bank. The system is able to perform its function as a versatile musical instrument. It is relatively stable, but needs a few revisions before it can be a reliable studio or stage performance-ready instrument.
	
	\subsubsection{What was learned?}
	
	\paragraph{The device platform must be chosen carefully} I chose the audio development board Daisy by Electro-Smith. It is marketed as an attractive option for musicians to implement their dream systems because of its high resolution digital to analog (DAC) converter and painless hardware peripheral abstractions. It turned out to not be as mature and stable as I hoped and had many issues with drivers, pin connections and lacking support for certain interrupt service routines. At the time, it was difficult to see that Daisy may not have been the best option for implementing system scoped beyond a hobbyist project. The platform was designed for musicians to implement simpler systems with highly abstracted development libraries and programming languages/frameworks like MaxMSP, Puredata and Arduino. It appears more attention and care was put into supporting these tools and making them work well while some lower level capabilities were sidelined. The Daisy platform is very popular and is used commercially but still has yet to reach a highly stable and mature state.
	
	\paragraph{Hardware prototyping must be incremental} After I had spent several months setting up my hardware peripherals on a breadboard and implementing core functionality, I felt the need to migrate the system off a breadboard to a more permanent and stable home. It was at this point I tried to do too much at once. I tried to design a soldered circuit board able to be embedded in a hardware enclosure that was yet to be designed. My thought was that it may take several revisions of the enclosure before I knew where all my knobs, encoders, inputs, outputs and screen needed to be for ease of use and optimal functionality. I built up flying wire harnesses for all peripherals so that they could be moved around to test out different enclosure layouts. The result was an absolute mess of wires that were susceptible to interference. My first step in taking the system off the breadboard should have been simply solder my components into logical sections on perf board and make the system work as it did on the breadboard. It would not have been optimal or have fit into an easy enclosure. I would still have to design a final enclosure and migrate the system to its final home, but I would have saved so much time soldering and troubleshooting my sloppy design. I could have allocated more time to software design and would have been able to complete more features like the LFO and patch bank mechanic.
	
	\paragraph{Pure software functions should be decoupled from hardware and tested independent of the hardware} My workflow for the entirety of the project was \textbf{write code, compile, flash, test, repeat}. 
	
	
	
	\subsubsection{Complexity}
	

\section{Related Works}
For each reference in your paper, write a two- or three-sentence paragraph 
describing how it relates to your work. Think of this as an annotated bibliography
covering your literature search in an organized progression that highlights your 
how your work is similar and different. This is particularly important if your project repeats
previous work. At least 15 sources is a reasonable number.

\textbf{Some authors prefer to include related works as part of the Introduction, because there
is overlap. For project defense purposes, it best to have it as a separate section.}

\section{Concepts}
This section is \textbf{optional}. Include it if you have ideas that a reader needs to know to
understand what you did, but which is too much for either Introduction and does not
fit naturally in other sections.

\section{Software Architecture and Implementation}
Describe here the design and implementation of your system. \textbf{This section is the heart of your paper.} Include figures that illustrate structure and behavior, especially critical aspects of the design and implementation. You should not detail every aspect of the project, just the most important information. Include:
\begin{enumerate}
    \item Meta-Architecture: Choices that influence your design and implementation, but are not those.
    \item Conceptual Architecture Diagram: A one-page "brochure-style" drawing of the concept and vision of your project.
    \item Structure and Behavior diagrams for architecturally significant parts of your system.
    \item screen shots
    \item example inputs/outputs for important scenarios
    \item key aspects of the implementation
    \item factual, objective discussion of implementation successes and challenges
\end{enumerate}
You may decide to discuss Implementation/coding issues in a separate section, but that is up to you.

\section{Project Results}
The Results section is a technical assessment of your project, keeping in mind that it is foremost a learning experience.
\begin{enumerate}
    \item What successes did you have?
    \item What failures?
    \item What was completed?
    \item What was planned but not completed?
    \item Looking back, what early choices were good or bad?
    \item What performance issues or bottlenecks exist in your project?
\end{enumerate}
Important things to keep in mind:
\begin{enumerate}
    \item An incomplete project is still a success if you learned from it.
    \item Failure is as important a result as success when doing research. The Wright Brothers did not fly 
    on their first attempt, but they did not give up.
\end{enumerate}

\section{Conclusion}
The conclusion should address the following items:
\begin{enumerate}
    \item Summarize and synthesize your learning and effort on the project.
    \item What is your personal view on the success or failure of the project?
    \item What did you learn in creating the project?
    \item What would have been helpful to know before starting the project?
    \item What was your most important results and conclusions?
    \item What are potential next steps for the project?
\end{enumerate}
The rest of the paper needs be factual and objective in tone, but in the conclusion you are free to express your
opinions, beliefs and feelings.

\section{Acknowledgement}
This section is optional.

\section{Figures and Captions}
\textbf{Remove this section in the finished paper.}
Figures have to be legible to be useful. LaTeX will automatically
number figures. In Word you have to do that by hand.

Captions should tell the user what they are seeing and why. Don't make the reader
hunt in the text for basic information.

\section{Tables}
\textbf{Remove this section in the finished paper.}
Format tables in nice columns with headings. Use only horizontal lines,
no vertical lines. LaTeX will automatically number tables.
In Word you have to do that by hand.

\section{References}
References should follow ACM format style.

Using BibTeX with LaTeX for preparing and formatting references 
is strongly recommended because it does the right thing for you.
In Word you have to manage references and citations by hand or use 
a third-party tool.

%%
%% The next two lines define the bibliography style to be used, and
%% the bibliography file.
\bibliographystyle{ACM-Reference-Format}
\bibliography{sample-base}

%%
%% If your work has an appendix, this is the place to put it.
\appendix

\section{User's Manual}
The User’s Manual section contains instructions for installing and running the project. Give sufficient detail that someone other than the candidate could install and use the software if they had prior knowledge of the domain.  \textbf{This User's Manual is a critical section of the paper and you cannot pass your defense without it.}
\end{document}
\endinput
%%
%% End of file `sample-manuscript.tex'.

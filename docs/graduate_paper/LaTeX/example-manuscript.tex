%%
%% This is file `sample-manuscript.tex',
%% generated with the docstrip utility.
%%
%% The original source files were:
%%
%% samples.dtx  (with options: `manuscript')
%% 
%% IMPORTANT NOTICE:
%% 
%% For the copyright see the source file.
%% 
%% Any modified versions of this file must be renamed
%% with new filenames distinct from sample-manuscript.tex.
%% 
%% For distribution of the original source see the terms
%% for copying and modification in the file samples.dtx.
%% 
%% This generated file may be distributed as long as the
%% original source files, as listed above, are part of the
%% same distribution. (The sources need not necessarily be
%% in the same archive or directory.)
%%
%% The first command in your LaTeX source must be the \documentclass command.
%%%% Small single column format, used for CIE, CSUR, DTRAP, JACM, JDIQ, JEA, JERIC, JETC, PACMCGIT, TAAS, TACCESS, TACO, TALG, TALLIP (formerly TALIP), TCPS, TDSCI, TEAC, TECS, TELO, THRI, TIIS, TIOT, TISSEC, TIST, TKDD, TMIS, TOCE, TOCHI, TOCL, TOCS, TOCT, TODAES, TODS, TOIS, TOIT, TOMACS, TOMM (formerly TOMCCAP), TOMPECS, TOMS, TOPC, TOPLAS, TOPS, TOS, TOSEM, TOSN, TQC, TRETS, TSAS, TSC, TSLP, TWEB.
% \documentclass[acmsmall]{acmart}

%%%% Large single column format, used for IMWUT, JOCCH, PACMPL, POMACS, TAP, PACMHCI
\documentclass[acmlarge,screen]{acmart}

%%%% Large double column format, used for TOG
%\documentclass[acmtog, authorversion]{acmart}

%%%% Generic manuscript mode


%%
%% \BibTeX command to typeset BibTeX logo in the docs
\AtBeginDocument{%
  \providecommand\BibTeX{{%
    \normalfont B\kern-0.5em{\scshape i\kern-0.25em b}\kern-0.8em\TeX}}}


%%
%% The majority of ACM publications use numbered citations and
%% references.  The command \citestyle{authoryear} switches to the
%% "author year" style.
%%
%% If you are preparing content for an event
%% sponsored by ACM SIGGRAPH, you must use the "author year" style of
%% citations and references.
%% Uncommenting
%% the next command will enable that style.
%%\citestyle{acmauthoryear}

%%
%% end of the preamble, start of the body of the document source.
\begin{document}

%%
%% The "title" command has an optional parameter,
%% allowing the author to define a "short title" to be used in page headers.
\title{UVU MCS Graduate Paper Template}

%%
%% The "author" command and its associated commands are used to define
%% the authors and their affiliations.
%% Of note is the shared affiliation of the first two authors, and the
%% "authornote" and "authornotemark" commands
%% used to denote shared contribution to the research.

\author{Candidate First Last}
\affiliation{%
 \institution{Candidate}
}
\email{candidate.email@uvu.edu}

\author{Advisor First Last}
\affiliation{%
 \institution{Advisor}
}
\email{advisor.email@uvu.edu} 

%%
%% By default, the full list of authors will be used in the page
%% headers. Often, this list is too long, and will overlap
%% other information printed in the page headers. This command allows
%% the author to define a more concise list
%% of authors' names for this purpose.
\renewcommand{\shortauthors}{Candidate First Last}

%%
%% The abstract is a short summary of the work to be presented in the
%% article.
\begin{abstract}
  Your abstract here: 250 words or less
  \begin{enumerate}
      \item What problem are you trying to solve?
      \item why is it interesting?
      \item What are your major results?
      \item What are your conclusions?
      \item What did you learn?
      \item How many lines of code is your project?
  \end{enumerate}
The abstract is a separate document from the rest of the paper and appears in indices separate from
the paper. In print, it always appears at the top of a paper. You may feel that the Abstract
repeats information in the Introduction, and it does--but on purpose.
\end{abstract}

%%
%% Keywords. The author(s) should pick words that accurately describe
%% the work being presented. Separate the keywords with commas.
\keywords{datasets, neural networks, gaze detection, text tagging}

%%
%% This command processes the author and affiliation and title
%% information and builds the first part of the formatted document.
\maketitle

\section{Introduction}
Begin with a proper thesis statement, which may be several paragraphs. A proper thesis statement
answers the following 4 questions:
\begin{enumerate}
      \item What problem are you trying to solve?
      \item why is it interesting?
      \item What are your major results?
      \item What are your conclusions?
\end{enumerate}
This is a technical paper, so you need to provide background and evidence for any claims you
make about your work and others' work, with citations.

It is helpful to your advisory committee if you also address these
two questions briefly. You will say more about these topics in other sections of the paper.
\begin{enumerate}
      \item What did you learn?
      \item How many lines of code is your project?
      \item If not lines of code, what makes your project complex?
\end{enumerate}

The size and scope of your project should be large enough to require 
5000 lines of code and take 10-20 pages to document. If you need help with writing, 
contact the UVU Writing Center.

You may use any editing software you want to produce your paper, but you must follow this 
template, and the final version you submit must be a PDF. In case you're interested, this template 
is a customized version of the ACM Sample Manuscript Template. We recommend using either 
MSWord or LaTeX for writing.

The last paragraph of a good Introduction outlines the sections in the
rest of the paper: One sentence per section. Guide the reader, don't surprise them.

\section{Related Works}
For each reference in your paper, write a two- or three-sentence paragraph 
describing how it relates to your work. Think of this as an annotated bibliography
covering your literature search in an organized progression that highlights your 
how your work is similar and different. This is particularly important if your project repeats
previous work. At least 15 sources is a reasonable number.

\textbf{Some authors prefer to include related works as part of the Introduction, because there
is overlap. For project defense purposes, it best to have it as a separate section.}

\section{Concepts}
This section is \textbf{optional}. Include it if you have ideas that a reader needs to know to
understand what you did, but which is too much for either Introduction and does not
fit naturally in other sections.

\section{Software Architecture and Implementation}
Describe here the design and implementation of your system. \textbf{This section is the heart of your paper.} Include figures that illustrate structure and behavior, especially critical aspects of the design and implementation. You should not detail every aspect of the project, just the most important information. Include:
\begin{enumerate}
    \item Meta-Architecture: Choices that influence your design and implementation, but are not those.
    \item Conceptual Architecture Diagram: A one-page "brochure-style" drawing of the concept and vision of your project.
    \item Structure and Behavior diagrams for architecturally significant parts of your system.
    \item screen shots
    \item example inputs/outputs for important scenarios
    \item key aspects of the implementation
    \item factual, objective discussion of implementation successes and challenges
\end{enumerate}
You may decide to discuss Implementation/coding issues in a separate section, but that is up to you.

\section{Project Results}
The Results section is a technical assessment of your project, keeping in mind that it is foremost a learning experience.
\begin{enumerate}
    \item What successes did you have?
    \item What failures?
    \item What was completed?
    \item What was planned but not completed?
    \item Looking back, what early choices were good or bad?
    \item What performance issues or bottlenecks exist in your project?
\end{enumerate}
Important things to keep in mind:
\begin{enumerate}
    \item An incomplete project is still a success if you learned from it.
    \item Failure is as important a result as success when doing research. The Wright Brothers did not fly 
    on their first attempt, but they did not give up.
\end{enumerate}

\section{Conclusion}
The conclusion should address the following items:
\begin{enumerate}
    \item Summarize and synthesize your learning and effort on the project.
    \item What is your personal view on the success or failure of the project?
    \item What did you learn in creating the project?
    \item What would have been helpful to know before starting the project?
    \item What was your most important results and conclusions?
    \item What are potential next steps for the project?
\end{enumerate}
The rest of the paper needs be factual and objective in tone, but in the conclusion you are free to express your
opinions, beliefs and feelings.

\section{Acknowledgement}
This section is optional.

\section{Figures and Captions}
\textbf{Remove this section in the finished paper.}
Figures have to be legible to be useful. LaTeX will automatically
number figures. In Word you have to do that by hand.

Captions should tell the user what they are seeing and why. Don't make the reader
hunt in the text for basic information.

\section{Tables}
\textbf{Remove this section in the finished paper.}
Format tables in nice columns with headings. Use only horizontal lines,
no vertical lines. LaTeX will automatically number tables.
In Word you have to do that by hand.

\section{References}
References should follow ACM format style.

Using BibTeX with LaTeX for preparing and formatting references 
is strongly recommended because it does the right thing for you.
In Word you have to manage references and citations by hand or use 
a third-party tool.

%%
%% The next two lines define the bibliography style to be used, and
%% the bibliography file.
\bibliographystyle{ACM-Reference-Format}
\bibliography{sample-base}

%%
%% If your work has an appendix, this is the place to put it.
\appendix

\section{User's Manual}
The User’s Manual section contains instructions for installing and running the project. Give sufficient detail that someone other than the candidate could install and use the software if they had prior knowledge of the domain.  \textbf{This User's Manual is a critical section of the paper and you cannot pass your defense without it.}
\end{document}
\endinput
%%
%% End of file `sample-manuscript.tex'.
